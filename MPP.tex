\documentclass{article}
\usepackage[colorlinks=true,linkcolor=blue,citecolor=blue]{hyperref}
\usepackage{url}
\usepackage[height=8in]{geometry}
\begin{document}
\begin{center}
\Large{Object-oriented Programming vs. Procedural Programming}\\
\smallbreak
\large{Fintan Hegarty}\\
\smallbreak
\large{G00376582}
\end{center}
\bigskip
\textbf{Abstract:} \textit{In this report, we examine the differences between object-oriented programming and procedural programming.}
\tableofcontents
\section{Introduction}
Computer programming, from the Latin \emph{programma}, meaning to provide a computer with a set of instructions to perform a specific task, is often considered to date back to Ada Lovelace's algorithm \cite{wikicp} whereby Charles Babbage's then-unbuilt ``analytical engine'' could calculate a sequence of Bernoulli numbers. We will not delve into the semantics; the interested reader may refer to, e.g., \cite{twobithistory}. The algorithm did use such constructs as variables and repeating loops, which are fundamental programming concepts. Later, Herman Hollerith's work \cite{colombia} led to a system for storing algorithms as well as data on the ``machine'', so that these could be recalled when required for computation, rather than needing to be input for each use.

This sets the foundation for \emph{procedural programming}, which we shall investigate briefly, and then move on to contrast this with another programming paradigm, called \emph{object-oriented programming}.

\subsection{What is procedural programming?}
% Examples of each type
A procedural programming language follows, in order, a set of commands. They use functions, conditional statements, and variables to enable a computer to calculate and display a desired output, or perform a certain task. 

\texttt{C} is a procedural programming language, which we have used in our research.  
\texttt{Python} is a multi-paradigm programming language, and can also handle object-oriented programs, but for this assignment, we used it as a procedural language. Other popular languages with procedural programming functionality include \texttt{ALGOL}, \texttt{Fortran} and \texttt{Pel}.
\subsection{What is object-oriented programming?}
% Examples of each type


\section{Similarities between object-oriented and procedural programming}
\subsection{Similarities in the exercise}

\section{Differences between object-oriented and procedural programming}
\subsection{Differences in our exercise}
% Example of code from both doing the same thing
% Juxtaposing table

\section{Conclusions}
\begin{table}
\begin{tabular}{c|c|c}
& Procedural Programming & Object-oriented Programming \\\hline
& & \\
\end{tabular}
\end{table}

\section*{Acknowledgements}
The author is grateful to Dr Dominic Carr for instruction in multi-paradigm programming, and to Microsoft Stream for inspiring him to strive for Zen-like levels of patience. 

\bibliographystyle{alpha}
\bibliography{MPP}
\end{document}
